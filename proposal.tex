% !TEX TS-program = pdflatex
% !TEX encoding = UTF-8 Unicode

% This is a simple template for a LaTeX document using the "article" class.
% See "book", "report", "letter" for other types of document.

\documentclass[11pt]{article} % use larger type; default would be 10pt

\usepackage[utf8]{inputenc} % set input encoding (not needed with XeLaTeX)
\usepackage{hyperref}
\hypersetup{
    colorlinks=true,
    linkcolor=blue,
    filecolor=magenta,      
    urlcolor=blue,
}
 
\urlstyle{same}


%%% PAGE DIMENSIONS
\usepackage{geometry} % to change the page dimensions
\geometry{a4paper} % or letterpaper (US) or a5paper or....
% \geometry{margin=2in} % for example, change the margins to 2 inches all round
% \geometry{landscape} % set up the page for landscape
%   read geometry.pdf for detailed page layout information

\usepackage{graphicx} % support the \includegraphics command and options

\usepackage[dvipsnames]{xcolor}
\usepackage{colortbl}

% \usepackage[parfill]{parskip} % Activate to begin paragraphs with an empty line rather than an indent

%%% PACKAGES
\usepackage{booktabs} % for much better looking tables
\usepackage{array} % for better arrays (eg matrices) in maths
\usepackage{paralist} % very flexible & customisable lists (eg. enumerate/itemize, etc.)
\usepackage{verbatim} % adds environment for commenting out blocks of text & for better verbatim
\usepackage{subfig} % make it possible to include more than one captioned figure/table in a single float
% These packages are all incorporated in the memoir class to one degree or another...

%%% HEADERS & FOOTERS
\usepackage{fancyhdr} % This should be set AFTER setting up the page geometry
\pagestyle{fancy} % options: empty , plain , fancy
\renewcommand{\headrulewidth}{0pt} % customise the layout...
\lhead{}\chead{}\rhead{}
\lfoot{}\cfoot{\thepage}\rfoot{}

%%% SECTION TITLE APPEARANCE
\usepackage{sectsty}
\allsectionsfont{\sffamily\mdseries\upshape} % (See the fntguide.pdf for font help)
% (This matches ConTeXt defaults)

%%% ToC (table of contents) APPEARANCE
\usepackage[nottoc,notlof,notlot]{tocbibind} % Put the bibliography in the ToC
\usepackage[titles,subfigure]{tocloft} % Alter the style of the Table of Contents
\renewcommand{\cftsecfont}{\rmfamily\mdseries\upshape}
\renewcommand{\cftsecpagefont}{\rmfamily\mdseries\upshape} % No bold!

\usepackage{pdfpages}

%%% END Article customizations

%%% The "real" document content comes below...

\title{CS 5630/6630 Project Proposal}
\author{Jon Bown \\ Pablo Napan \\ Andrew Yang}
\date{} % Activate to display a given date or no date (if empty),
         % otherwise the current date is printed 

\begin{document}
\maketitle

\section{Basic Info}
	\subsection{Project Info}
	\textbf{Project Title:} TubeVis \\
	\textbf{Github Repository:} \url{https://github.com/GuitarPro6/dataviscourse-pr-TubeVis}
	\subsection{Jon Bown}
	\textbf{Email:} jonvzw6@gmail.com\\
	\textbf{UID:} u0696785 
	\subsection{Pablo Napan}
	\textbf{Email:} pablo.napan@utah.edu\\
	\textbf{UID:} u1077731
	\subsection{Andrew Yang}
	\textbf{Email:} u0778110@utah.edu \\
	\textbf{UID:} u0778110

\section{Background and Motivation}
This is something that hasn't really been done (and should be done), yet also achievable within the scope of this course. There weren't any research interests or backgrounds that led to us choosing these data sets.

\section{Project Objectives}
We wanted to answer these questions:
\begin{itemize}
\item What are the changes over time for London's underground tube stations (usage, demographic, fares, etc)?
\item What does the demographic that use these stations look like?
\item Where do people come from and go that use these stations?
\end{itemize}
We will gain a better understanding of London's public transportation system and improve our coding skills at the same time through the making of this project.

\section{Data}
The data sets we are using comes from \url{https://data.london.gov.uk} and \url{https://api-portal.tfl.gov.uk}. They include the entries and exits of London underground stations, surveys that showcase some background information about the users of those stations, and more statistics about London's underground transportation system over the past decade.

\section{Data Processing}
The data sets we will be using are already well organized. We just have to break up spreadsheets into parsable data for Javascript and D3. We will beexternal libraries for this process. The proposed libraries are from Github user \href{https://github.com/sheetjs}{Sheet JS}.

\section{Visualization Design}
See Appendix

\section{Must-have Features}
\begin{itemize}
\item See whole map, a certain tube, and specific stations, along with corresponding data
\item Stretch a irregularly shaped tube into a straight line when clicked on
\item Compare two difference stations/lines
\item Filter whole map to show aggregate statistics with a selector
\item Show entering and exiting data per line and per station
\item Sliders showcasing data for a station based on time of day 
\item Interactive objects if appropriate 
\end{itemize}

\section{Optional Features}
\begin{itemize}
\item Networking graph with clickable objects (see sketches) 
\item Time lapse based on what time of day it is for the whole line
\item Chord design for common routes taken by London citizens
\end{itemize}

\section{Project Schedule}
\begin{center}
\begin{tabular}{c|c}
Time & Goal \\
\hline
\hline
October &  \\

Week 3 & Rendering a map \\
Week 4 & Plugging in data to the map and building utilities \\
\hline
November &  \\
Week 1 & Focus on individual tubes/lines\\
Week 2 & Focus on stations, visualize demographic/station information\\
Week 3 & Additional features not yet completed\\
Week 4 & Optional features if time permits\\
\hline
December & \\
Week 1 & \\
Week 2 & \\
\end{tabular}
\end{center}

\section*{Appendix}

\includepdf[pages=-]{Project_Proposal_Appendix.pdf}

\end{document}
